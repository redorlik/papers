\section{Conclusion}



In this multidisciplinary study I have learned from many different fields. I have learned from social and political scientists about how to assess climate change policy and the processes in the United Nations organisations devoted to climate change policy design. From chemistry I have learned about atmospheric chemistry and how to model dispersion of airborne pollutants. I have used signal analysis, and knowledge on digital signal processing to interpret measured data. From Environmental science I have learned detailed methodology for estimating emission of pollutants and from computer science I have learned about mobile sensing, database design and Enterprise Architecture.
The first 2 years of the phd study has primarily been used to learn about the fields, and find ways to apply the knowledge in the context of the EcoSense project.
The main scientific contributions are
\begin{enumerate}
	\item{Coauthoring a paper on testing of Semantic services}
	\item{Development and implementation of a simple model for estimation of single trip emissions}
	\item{Development of more precise model for emission estimation, using actual vehicle speed, instead road type, as emission factor}
\end{enumerate}
The future work as described in section \ref{futurework}, will be focused on filling the gap identified in the methodology for estimating the emission of airborne pollutants from traffic, and evaluating the above listed contributions. The gap exists between the precise methods of measuring the  emissions directly from the tailpipe, by instrumenting the vehicle with extra sensors, and the use of average values for activity data and engine data. With mobile sensing the activity data can be more accurate than the mean values estimated for activity data in COPERT IV and IPCC methodologies.

The future improvement in single trip estimation of emission of pollutants, will be based on more precise measurements of the four driving modes : Idle, acceleration, cruising and deceleration. To get more precise information on these driving modes, different combinations of the data from the different sensors will be used.

The search for route patterns will be performed, when data from the field trials, will be available. To prepare for this work, I will participate in the course "Route choice models" at DTU Transport in fall of 2014.

Extensions to the pollution dispersion models, with respect to new chemical reactions, is planned for the coming study period.

To sum up the the possible impact of this phd study:

\begin{enumerate}
	\item Improve green accounting for passenger transportation.
	\item Facilitate building of regional and local emission inventories.
	\item Improve the calculation of the impact on Climate Change from passenger transportation.
	\item Enable air pollution guided traffic planning, due to better description of local air pollution.
\end{enumerate} 
I find that the possibility to be able to create emission inventories for a municipality or a company , even if there is no complete coverage of mobile sensing devices, to be quite interesting, and I look forward to further improve the methods describe in this progress report.

\addsec{Acknowledgement}
I would like to thank my supervisors Niels Olof Bouvin and Allan Gross, as well as Henrik Blunck for the interest, comments, suggestions and great engagement in my phd. study. 

A big thank you goes to Markus W\"ustenberg, for providing the rich dataset, the study is based on.

I would also like to thank my family for their patience and support.

\vspace{20pt}
This work has been supported by the Danish Council for Strategic Research as part of the EcoSense project (11-115331).