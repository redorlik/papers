\section{Climate change policy development}
\subsection{Climate change organisations}

\subsection{Climate change mitigation}From the start of the debate about how to combat Anthropogenic Global Warming, there have been two competing approaches. The mitigation approach, where an effort is made for reducing the emissions of climate gases, is the approach which has received the most attention. The EcoSense 3 project is part of the mindset behind this approach. Whereas the mitigation approach previously has focused on creating more efficient machines, to either produce energy with less emissions or produce machines that produce more useful work per energy unit, EcoSense focuses on how the machines are used in combination. So instead of focusing on a single energy consuming item, we are trying to analyze how the energy consumption reacts as the individual items work together. In other words we are developing tools for studying networks effects in the transportation sector, as well as in other sectors.


\subsection{Climate change adaptation}
The second approach is called the Adaptation approach. Since it is probable that there will be significant changes in the climate, even if we succeed in keeping the CO2 emissions within the agreed limits, we have to invent ways to adapt to these changes. The Adaptation approach has not received much attention or research funding, but projects like EcoSense will also have an impact in determining how to adapt to changes in weather patterns. It seems that IPCC in the 5. report from Working Group 2 is beginning to give more attention the the Adaptation approach, as a consequence of realizing that some level of Climate Change is inevitable.

