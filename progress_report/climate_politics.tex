\section{Climate change policy development}\label{politics}
\subsection{Climate change organisations}
It has been decided in the United Nations to create two bodies for devising policies for managing the anthropogenic global warming. The main political body is the United Nation Framework Convention for Climate Change (UNFCCC) \cite{schipper2009}. The name refers to both a convention created in 1994, and a secretariat that services the operation of the Convention. This body is tasked with creating policy initiatives and negotiate treaties with the members of the UN. A visible part of these negotiations are the COP (Conference of parties) meetings held annually. Since 154 nations have ratified the convention, the negotiations are very difficult, and progress in for instance agreeing on firm goals and plans to reach the goals is slow \cite{pielke1998}.

To support the UNFCCC with scientific knowledge the Intergovernmental Panel on Climate Change (IPCC) was created. The panel gathers scientific evidence into reports on different aspects of climate change, the root causes, the consequences and ways to mitigate and adapt.

Even though the panel is a scientific panel, it is also a part of a political process. This means that the reports made by IPCC, especially the "summary for policy makers" are also heavily negotiated by the participating government officials.

The latest reports from IPCC are the AR-5 (fifth assessment report) series .

The IPCC reports are divided into sections, done by different workgroups. Workgroup 1 delivers a scientific background section containing the latest scientific knowledge about emissions, measured consequences, melting of Icecaps etc. As a new feature in the assessment report a number of scenarios are envisioned, to estimate the consequences of different policies. The scenarios are called "Representative Concentration Pathways", and are named by the expected rise in global mean temperature. Four scenarios are envisioned in the report: RCP2.5, RCP4.5, RCP6, RCP8,5 \cite{stocker2013climate}. These scenarios are then used as a guiding principle for the rest of the sections in the AR-5.

The workgroup 1 report is generally viewed as a high quality and reliable source of knowledge.

Workgroup 2 delivers a report on "Impacts, Adaptation and Vulneralbility". 

Workgroup 3 is about mitigation of climate change.



\subsection{Climate change mitigation}From the start of the debate about how to combat Anthropogenic Global Warming, there have been two competing approaches. The mitigation approach, where an effort is made for reducing the emissions of climate gases, is the approach which has received the most attention\cite{Pielke2007}. The EcoSense project is part of the mindset behind this approach. Whereas the mitigation approach previously has focused on creating more efficient machines, to either produce energy with less emissions or produce machines that produce more useful work per energy unit\cite{Chakraborty2013}, EcoSense focuses on how the machines are used in combination. So instead of focusing on a single energy consuming item, we are trying to analyse how the energy consumption changes as the individual items work together. In other words we are developing tools for studying network effects in the transportation sector, as well as in other sectors.


\subsection{Climate change adaptation}
The second approach is called the Adaptation approach. Since it is probable that there will be significant changes in the climate, even if we succeed in keeping the $CO_2$ emissions within the agreed limits, we have to invent ways to adapt to these changes. The Adaptation approach has not received much attention or research funding, but projects like EcoSense will also have an impact in determining how to adapt to changes in weather patterns. It seems that IPCC in the fifth report from Working Group 2 is beginning to give more attention the Adaptation approach, as a consequence of realising that some level of Climate Change is inevitable \cite{kelly2000}\cite{smithers1997}.

