\section{Introduction}
During the last 40 years an overwhelming amount of evidence for Anthropogenic Global Warming has been gathered [9]. This evidence has led to international agreement on certain limits for the emission of gases that affect the global warming, most notably CO2. In order to reach these goals and keep the level of CO2 within the agreed limits, technological solutions need to be developed, partly to develop cleaner energy production technology, and partly to develop more energy efficient ways of living. Previously the focus has been on developing more efficient machines, under the assumption that the use patterns of the energy consuming machines were optimal.

The research project EcoSense is aimed at analysing user behaviours, and if possible give advice to change user behaviour to be more resource efficient.

One driving goal for the project, is to be able to improve the accuracy and ease of use of green accounting for companies. Today it is a tedious and time-consuming task to estimate the amount of emissions a company is responsible for. One of the cumbersome areas of green accounting is with regard to emissions due to transport. There is a need for methods and models that will account for the actual emissions due to the day to day business of the company and not accounting done from mean values.

My contribution to the project is threefold:
	1) to create and improve emission models for individual trips.
	2) to create models for aggregation and correlations of multiple trips.
	3) to improve models for the dispersion of emissions.

By succeeding in these goals the impact would be:
	Improved accuracy of green accounting
	Providing hints to increase efficiency of transportation
	
In the first part of this report I will go through the work done so far in the study, and give a background to understand the methods used.

The section \ref{futurework} is a plan for the remaining period of the study.

\alnote{More on EcoSense
motivation
Reading guide
}
