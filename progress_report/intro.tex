\section{Introduction}
During the last 40 years an overwhelming amount of evidence for Anthropogenic Global Warming has been gathered \cite{schipper2008earthscan}. This evidence has led to international agreement on certain limits for the emission of gases that affect the global warming, most notably $CO_2$.  The gases that affect the climate through warming of the atmosphere, are commonly known as radiative forcing gases or climate forcing gases. In order to reach these goals and keep the level of $CO_2$ within the agreed limits, technological solutions need to be developed, partly to develop cleaner energy production technology, and partly to develop more energy efficient ways of living. Previously the focus has been on developing more efficient machines, under the assumption that the usage patterns of the energy consuming machines were optimal.

This phd study is undertaken as part of the three year research project EcoSense\footnote{http://ecosense.au.dk} which is aimed at analysing user behaviours, through the use of mobile sensing with smartphones, and if possible give advice to change user behaviour to be more resource efficient. 
EcoSense is not the only project trying to use mobile sensing \cite{mun_peir_2009} \cite{calabrese1977}. My part of EcoSense is to develop,  implement and evaluate models for emissions. In this report I will mostly refer to $CO_2$ emissions as this is the most important climate forcing gas. To measure the total impact on climate forcing, the contributions of other gases has to be taken into account. This is done by converting the contributions into $CO_2$ equivalents - that is the amount of $CO_2$ giving the same amount of radiative forcing.

When modelling emissions of climate gases from passenger transportation, techniques for modelling emissions of air pollutants are employed, which enables me to also study air pollution from passenger transport. Air pollution has different aspects, compared to climate science, mainly because the pollutants are considered due to their toxicity properties (towards humans) instead of their radiative forcing properties.

Due to the complexity and multifaceted nature of emission modelling the study has become a multidisciplinary study. That is, I have used methods, knowledge and technologies from different scientific fields such as computer science, environmental science, climate science, political science and engineering science.

\subsection{Problem Statement}
One driving goal for the project, is to be able to improve the accuracy and ease of use of green accounting for companies. Today it is a tedious and time-consuming task to estimate the amount of emissions a company is responsible for. One of the cumbersome areas of green accounting is with regard to emissions due to transport. There is a need for methods and models that will account for the actual emissions due to the day to day business of the company and not accounting done from mean values. In the study the focus has been on passenger cars, since passenger cars is responsible for about 60\% of the emission of green house gases from transportation in Denmark \cite{nielsen2014}. The methods studied will be transferrable to other vehicle types.

In the EcoSense project the emphasis is on climate gases, but since emissions from passenger transport also contains other pollutants with no or small climate forcing properties, the methods for modelling the emissions of climate gases can also be used for modelling the emissions of pollutants. Thus I will investigate how the mobile sensing data of passenger transport can be exploited to improve modelling of air pollution.

Another focus of the study is to use the amount of data collected to be able to help inform and inspire traffic planning.


\subsection{Methodology}
In the following I will discuss the methods used for estimating emission of climate forcing gases from transportation and proposals for improving these methods.

As the study is inspired of a number of different fields, spanning from Climate Change policy over atmospheric chemistry to data science and signal analysis, I have adapted a number of different methods for solving the problems. 

\subsubsection{Secondary research}
To find suitable models for emissions of Climate forcing gases from traffic a number of emission accounting methodologies has been investigated through a systematic literature search and appraised for the suitability of the methods in the EcoSense project.

\subsubsection{Data analysis}
The data that is gathered from smartphones, is essentially timestamped values from the sensors in the phones. To analyse the data it is first converted into a time series format before a number of different methods are applied in order to find and present interesting patterns. For analysing GPS traces the distance travelled is calculated by applying an algorithm taking the curvature of the earth into account, thus employing methods from geometry.

For analysing accelerometer data algorithms from signal analysis, are employed to explore the data and extract  information on engine speed and vehicle activity.
\subsubsection{Field trials}
The models developed in the first part of the phd study are going to be applied to field trials in different cities in Denmark, through smartphone applications designed for each trial. These field trials will be used as an evaluation of the used methods.

\subsection{Contributions}
To get information on the emissions from a single trip in a passenger car, the most precise way would be to measure the emissions from the tailpipe \cite{Frey}. This is a bit impractical as it involves instrumenting the vehicle.

As a more crude estimate, the methods from IPCC inventory guidelines can be used. These methods are designed to make the task of creating national climate forcing gas emission inventories, a practical endeavour. In between these two methods there is a gap, which I will try to fill, without extra instrumentation of the vehicles.

My contribution to the project is threefold:
\begin{enumerate}
  	\item To create and improve emission models for individual trips.
  	\item To create models for aggregation and correlations of multiple trips.
  	\item To improve models for the dispersion of emissions.
\end{enumerate}

By succeeding in these goals the impact would be:
\begin{enumerate}
  	\item 	Improved accuracy of green accounting.
	\item		Providing hints to increase efficiency of transportation.
\end{enumerate}
	
\subsection{Reading guide}
In the first part of this report ( section \ref{politics} to section \ref{IoT} ) I will go through the related work I have found, and give a background to understand the methods used. The second part of the report is describing my contributions and my planned future work. 

Section \ref{politics} is an overview of the international institutions put into place to oversee and develop Climate Change policies. It is important to keep in mind that the problems of Climate Change mitigation and adaptation are global problems, thus has to be solved in an international setting.

Section \ref{modelling} describes the work done so far to understand and implement models for estimating the emission of climate forcing gases from traffic. The methodology decided by IPCC is described and the way that the principles has been adopted to single trip estimation is elaborated.

Section \ref{dispersion} is about ways to model the movement of airborne pollutants. Originally these methods were developed for environmental planning purposes, but in this study it will be combined with the detailed knowledge of the emission pattern from traffic.

Section \ref{IoT} concerns ways connecting and relating data from sensors on different devices.

In the section \ref{contributions}, I have made a detailed description of the contributions I have made in the study to date.

The section \ref{futurework} is a plan for the remaining period of the study.

