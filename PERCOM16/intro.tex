\section{Introduction}

The proliferation of smartphones brings new opportunities to researchers who are studying the behaviour of modern human beings. For research in mobility and transport the proliferation of smartphones yields the potential to lower the cost for gathering relevant data to near zero \cite{Liu2013}.
In this project, smartphones are a central requisite as an tool information and data gathering tool. The main idea in the project is to gather from real life activities, create models to interpret the data, and visualisations from the output of the models. 

In modern societies, carbon emissions from transport are a large and growing part of the total carbon emissions from human activities. For instance, in Denmark the emission of carbon dioxide from transport was 24 \% of the total Carbon emissions in 2012 \cite{nielsen2014}, and road transport is responsible for 67 \% of the Carbon emissions from transport. In order to be able to create viable measures to lower the carbon emissions from passenger transport, there is a need for more detailed models quantifying these emissions. By using data from smartphones we aim to obtain more accurate information on where and when emissions from transport are created. These models can help policy makers and planners make informed decisions on future changes to the road transport system and related infrastructure.

As Carbon emissions are closely related to the fuel consumption, experiments were designed and carried out which gathered from an onboard smartphone  driving-related data and data on the amount of fuel consumed.

\todo{can be ommited?:}
To accurately model Carbon emissions from smartphone data, an experiment was performed in order to establish a ground truth\todo{ground truth: explain what and why?}. 

We chose to use some old car \todo{exact details? btw, use different cars for a broader range of experiments?} for the experiments and thus 
\todo{this sounds like a poor choice and excuse? Can we say that we aim for getting data also from all includung old cars? Btw, what about the instrumenting software that you had trouble with but which our developers now have instrumented on our department car? Would that make that your statement above obsolete?} had to create an accurate fuel consumption measuring method, which did not involve instrumenting the vehicles. 


The remained of the paper is organized as follows:
In Section \ref{sec:relwork} we review related work, including methods for measuring the fuel consumption and for creating related models of fuel consumption from sensor data  Additinoally, we briefly review methods to determine different driving styles and driving modes.
In Section \ref{sec:modeling} the models used to convert the experimental data into emission data are presented. 
The results of the experiments which partly relate to fuel consumption measurement and modeling and partly to reliable detection of driving modes, are presented in Section \ref{sec:results} and the paper is concluded in Section \label{sec:conclusions}.