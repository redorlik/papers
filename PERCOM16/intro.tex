\section{Introduction}

The proliferation of smartphones, brings new opportunities to researchers who are studying the behaviour of modern human beings. For research in mobility and transport smartphones potentially lowers the cost of gathering data for analysis to near zero \cite{Liu2013}.

In this project, smartphones are a central requisite as an information and data gathering tool. The main idea in the project is to gather from real life activities, create models to interpret the data, and visualisations from the output of the models. 

Carbon emissions from transport is a large and growing part of the total carbon emissions. In Denmark it was 24 \% in 2012 \cite{nielsen2014}. In order to be able to create viable measures to lower the carbon emissions from transport, there is a need to have more detailed models of the carbon emissions. By using data from smartphones it is possible to get more accurate information on where and when the emissions from tranport are created. 

To accurately model Carbon emissions from smartphone data, an experiment was performed in order to establish a ground truth. The Carbon emissions are closely related to the fuel consumption, so the experiment gathered data from a smartphone and data on the amount of fuel consumed in the experiment. We chose to use some old cars for the experiment and thus had to create an accurate fuel consumption measuring method, which did not involve instrumenting the vehicles. The experiment is described  in section three. To find a method for measuring the fuel consumption and creating models for , we studied the literature, as can be seen in section two.

In section four the models used to convert the experimental data into emission data is presented. The results from the research are presented in section five and the paper is concluded in section six.