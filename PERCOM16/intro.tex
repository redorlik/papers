\section{Introduction}

The proliferation of smartphones, brings new opportunities to researchers who are studying the behaviour of modern human beings. For research in mobility and transport smartphones potentially lowers the cost of gathering data for analysis to near zero \cite{Liu2013}.

Carbon emissions from transport is a large and growing part of the total carbon emissions from human activities. In Denmark the emission of carbon dioxide from transport was 24 \% of the total Carbon emissions in 2012 \cite{nielsen2014}, and road transport is responsible for 67 \% of the Carbon emissions from transport. In order to be able to create viable measures to lower the carbon emissions from passenger transport, there is a need to have more detailed models of the carbon emissions. By using data from smartphones it is possible to get more accurate information on where and when the emissions from transport are created. These models can help policy makers and planners make informed decisions on future changes to the road transport system.

To accurately model Carbon emissions from smartphone data, an experiment was performed in order to establish a ground truth. The Carbon emissions are closely related to the fuel consumption, thus the experiment gathered data about driving from an onboard smartphone and data on the amount of fuel consumed in the experiment. We chose to use some old car for the experiment and thus had to create an accurate fuel consumption measuring method, which did not involve instrumenting the vehicles. 

The experiment is described  in section three. To find a method for measuring the fuel consumption and creating models for the fuel consumption from the available smartphone data, we studied the literature. We also looked for method to determine different driving styles and driving modes. An overview of the field is presented in section two.

In section four the models used to convert the experimental data into emission data are presented. 

The results of the experiment, which is partly about fuel consumption measurement and modelling and partly about reliable detection of driving modes, are presented in section five and the paper is concluded in section six.