\section{Experiment}
An accurate model for Carbon emission from transport, can be derived from an accurate model of fuel consumption of vehicles, since the Carbon emissions are related to the combustion of fuel. 
To get a baseline for the fuel consumtption model, an experiment was performed. Data from a smartphone was combined with accurate measurements of fuel consumption. 

Researchers have reported fuel consumption measurements by instrumenting the vehicles with flow gauges in the fuel line. These measurements allow for realtime measurements of the fuel consumption. For this experiment flow meters were not available, nor was it possible to instrument the cars. Instead an elaborate measurement procedure was developed to attain accurate fuel consumption measurements.

\subsection{Route selection}
To select a route for the experiment, a number of criteria was considered. The length of the route should be long enough to have a measurable consumption, but short enough to be repeatable, and certainly less than the range of a tankful of fuel. The route should comprise different driving patterns (urban, rural, high speed/low speed) to simulate real life driving patterns. 

The route was chosen to mainly be in an urban traffic setting. The urban setting was chosen in order to get as many accelerations and idle periods a possible. The route did contain a short distance on a rural highway in order to have high speed measurements as well.

The length of the route was chosen to be 13 km, which was short enough to be repeated, and long enough to make measurement of fuel consumption possible. The fuel consumption for the route was approximately one liter of fuel, depending on driving style.

To prevent cold start effects to affect the tests, the vehicle is driven until the engine has reached the operating temperature before starting the test. The chosen course was run a number of times and the fuel consumption was determined after each run.

\subsection{Measurement of fuel consumption}

As it was not possible to install mass flow meters in the test vehicles the following procedure was developed to determine the fuel consumption. A the start of the course the gas tank was filled and  the fuel level in the gas tank funnel was measured. After the course was completed, gas was refilled so the same level of gas in the funnel, from a fuel canister. The weight of the fuel canister was measured before and after the gas filling process, to determine the amount of fuel refilled, and thus the consumption of fuel for the trip. To get a reproducible reading of the weight of the canister the weight has to be placed at the same place, with the same orientation, for each measurement. An ordinary kitchen scale with 1g accuracy was used in the experiment. In order to reduce the effect of wind on the measurement the scale and canister was placed in a cardboard box during the measurement.

This elaborate method was needed since the use of gas dispenser at the gas station, proved inaccurate. The main problem of using the fuel dispenser to measure the amount of fuel consumed driving the test route, is that it is hard to get the same filling level at each each filling. Using the automatic filling stop mechanism is unreliable, as the filling level becomes dependant on how much foam the filling process has created in the tank and the funnel. This creates large deviations of the amount of fuel to replenish the gas after the test runs. 

By accurately measure the filling level and refill the gas from a canister, made it possible to have very small differences between subsequent runs, with equal driving styles. 

The accuracy of the measurement can be estimated by inspecting the uncertainty of the individual steps of the measurement. The tank filling level can be measured with an accuracy of less than one cm, and since the diameter of the funnel is five cm the maximum error of the fuel measurement will be less than 20 cm\textsuperscript{3} or 20 ml, which is 2 \% if the total fuel refill is one liter.
